\documentclass[10pt,letter]{article}
    % basic article document class

\usepackage{fullpage}
    % package that specifies normal margins
\usepackage{makecell}
\usepackage{setspace}
\usepackage{tabularx}


\doublespacing

\begin{document}
    % line of code telling latex that your document is beginning

\title{CS341 Shopping Cart Project Plan}

\author{Sal Skare, Jack Englund, David Wissink, John Collins}

\maketitle 
    % tells latex to follow your header (e.g., title, author) commands.

\section*{Project Overview} The aim of this project is to develop a shopping cart software. A customer can browse or
search items, add an item to or remove an item from his/her shopping cart, and edit item’s
quantities in the shopping cart. The check out procedure requires promotion code if
applies, shipping address, tax calculation and credit card information. An administrator
can add an item to or remove an item from the system, and edit the item’s description and
quantity in the system. The software should also have promotion occasionally.
\\
The following minimal set of functionalities must be included:
\begin{itemize}
    \item Register a customer into the system.
    \item Modify the registered customer’s profile.
    \item Browse, search, add, remove or edit items in the system or shopping cart.
    \item Apply promotion code when check out if the promotion is valid.
    \item The tax rate will be calculated based on the credit card’s billing address.
    \item There should be at least three types of users: admin, registered customer, and guest.
\end{itemize}

\par A graphical user interface will be developed for this software. The interface must consist
of multiple screens for different tasks. For example, there should be a welcome screen to
start with which displays the various functionalities of the software. If one chooses to
login or register to the system, there will be a separate window/screen that displays the
requested information. 

\par Our approach to this project will be to create a Software-as-a-Service model of cart software. Site administrators will be able to tailor their frontend to their business needs, and access the endpoints on our server using JavaScript.

\section*{Project Plan}
\begin{tabularx}{\textwidth}{X | c | X | c | X }
    \hline
    Deliverable & Estimated time to complete & Actual time & Deadline & Notes \\
    \hline
    Functional Requirements, Use Case Models, User Stories, Assumptions & 1.5 hours & & Sept. 27 & \\
    \hline
    Demo 1 & 20 hours & & Oct. 11 & \\
    \hline
    UML Diagrams & 10 hours & & Nov 1. & \\
    \hline
    Demo 2 & 25 hours & & Nov 15. & \\
    \hline
    Updated Documentation & 10 hours & & Nov. 29  & \\
    \hline
    Final Demo & 30 hours & & Dec 11 & \\
    \hline
    Final source code & 2 hours & & Dec 11. & We aim to keep our source code clean and well commented through the whole project \\
    \hline
    User manual & 4 hours & & Dec. 11 & \\
    \hline
\end{tabularx}

\end{document}
